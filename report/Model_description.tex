\chapter{Descrizione del modello}
\label{chap:modeldesc}
Il sistema che verrà descritto in questo capitolo ha lo scopo di individuare nel minor tempo possibile tutte le persone ferite all'interno di un'area cittadina che si presuppone essere stata colpita da una qualche calamità naturale.
Per far ciò, si desidera che i robot esplorino l'intera mappa il più velocemente possibile.\\
La soluzione proposta è un adattamento di un noto lavoro in letteratura realizzato da Burgard \textit{et al.} \cite{burgard2005}, a cui si aggiunge un ulteriore capacità e compito dei robot: creare una rete \textit{wi-fi mesh} mediante il posizionamento di ripetitori realizzati \textit{ad hoc} per queste situazioni \todo{citare paper ibm}.
Tale rete possiede principalmente due scopi: \begin{itemize}
	\item permettere ai feriti non ancora individuati di segnalare la loro presenza se si trovano all'interno della copertura della rete, come proposto nell'articolo;
	\item permettere ai robot di comunicare liberamente tra loro, utilizzando quindi un approccio di comunicazione globale. Ciò permette una facile condivisione di informazioni e il mantenendo di una più o meno elevata forma di autonomia decisionale da parte del singolo agente.
\end{itemize} 
In questo capitolo, verrà dapprima descritto l'ambiente in cui gli agenti operano e cosa esso contiene da un punto di vista programmativo; verranno quindi descritte le due tipologie di agenti presenti all'interno del sistema, descrivendone il comportamento e, per quanto riguarda i robot, come ci si aspetta che essi comunichino tra loro.
\section{Ambiente}
\label{sec:environment}
L'ambiente in cui agiscono gli agenti è una porzione di territorio, tipicamente cittadino e di dimensioni variaibli, dopo che è avvenuto un evento catalogabile come disastro naturale; per questo motivo, porzioni di tale mappa non potranno essere esplorate, poiché inaccessibili e le parti esplorabili non richiederanno tutte lo stesso tempo di esplorazione a causa della complessità di attraversamento del territorio causato dai detriti.
L'ambiente presenta le seguenti caratteristiche:
\begin{itemize}
	\item parzialmente inaccessibile, perché i robot esplorano in modo progressivo il territorio; \textit{i.e.}, all'inizio non conoscono nulla della mappa se non quello che percepiscono con i loro sensori, man mano che l'esplorazione progredisce conoscono sempre una porzione maggiore del territorio fino a conoscere tutto l'ambiente ad esplorazione ultimata;
	\item stocastico, poiché nonostante il comportamento dei robot è ben definito a priori, quello dei feriti (il secondo tipo di agente) e le probabilità di fallimento dei robot o dei ripetirori \textit{wi-fi} sono stabiliti da regole stocastiche;
	\item sequenziale, le azioni dei singoli robot dipendono da quelle che hanno effettuato in precedenza, inoltre i feriti possono segnalare la loro presenza solo se non sono già stati individuati dai robot;
	\item semi-dinamico, i robot e i feriti non agiscono direttamente sull'ambiente modificando il territorio ma possono modificare l'importanza di sue porzioni, che come descritto in seguito, cambieranno il comportamento degli agenti;
	\item discreto, nonostante la combinatoria sia significativa è possibile stabilire a priori tutte le possibili configurazioni che può assumere l'ambiente con gli agenti al suo interno.
\end{itemize}

Come già accennato, l'ambiente è rappresentato come una griglia in cui ogni cella rappresenta un'area di 3$\times$3 metri e ogni \textit{step} della simulazione corrisponde ad un secondo di tempi di orologio.
Inoltre, attorno al territorio da esplorare, vi è un bordo composto da una “cornice di spessore una cella” che rappresenta la porzione di territorio confinante a quello di interesse in cui verranno disposti i robot, e che questi potranno sfruttare per raggiungere più velocemente altre celle all'interno dell'area di interesse.
Ogni cella è descritta da un insieme di attributi:
\begin{itemize}
	\item le sue coordinate all'interno della griglia, per rappresentazione interna del simulatore viene prima esplicitata la colonna e poi la riga;
	\item un intero che varia nell'intervallo $\\left[1, 12\\right]$ che rappresenta una difficoltà simbolica della cella, più tale valore è alto più il robot impiegherà tempo ad esplorarla e ad attraversarla per raggiungere altre celle (un difficoltà elevata può essere data da un numero maggiore di detriti nella zona o a dei muri che costringerebbero il robot ad effettuare a livello microscopico degli aggiramenti);
	\item lo stato della cella, ovvero se non è ancora stata esplorata, se sta venendo esplorata, se non è esplorabile oppure se è una cella della “cornice”;
	\item un valore di priorità, ovvero un parametro che fa aumentare l'importanza della cella favorendola nella scelta della prossima destinazione da parte dei robot, ciò è dovuto al fatto che la cella si trova nel vicinato di una cella in cui il ferito ha segnalato la sua posizione \todo[inline]{in che intervalli di valori varia? DP};
	\item un valore di utilità, inizilizzato ad uno per ogni cella, che viene sfruttato dai robot per scegliere la cella “migliore” (questo concetto viene meglio delineato nella Sotto-sezione \ref{sub:robots});
	\item due booleani che stabiliscono se in tale cella è stato posizionato un ripetitore \textit{wi-fi} oppure se la cella è coperta dal segnale \textit{wi-fi}, si sottolinea che la rappresentazione della copertura della rete è effettuata mediante questa tecnica e che non si sono sfruttati ulteriori agenti rendendo quindi tale rappresentazione a stretto contatto con l'ambiente.
\end{itemize}

Da un punto di vista programmativo, l'ambiente non rappresenta solo l'ambiente in sè ma si preoccupa di far progradire la simulazione, raccogliere i dati di interesse, contenere dei parametri utilizzati dai singoli robot (per comodità di rappresentazione dei dati e di gestione della memoria) e infine di fungere anche come parte della rappresentazione condivisa della mappa da parte dei robot, quest'ultima è discussa in dettaglio nella Sotto-sezione \ref{sub:robots}.
Di seguito, verranno elencati, e brevemente descritti, solo gli attributi che si riferiscono effettivamente all'ambiente:
\begin{itemize}
	\item \texttt{grid} rappresenta la griglia in cui i vari agenti si muoveranno;
	\item \texttt{schedule} rappresenta uno \textit{scheduler} con ordine di attivazione casuale per l'esecuzione dell'azione degli agenti al relativo \textit{step} della simulazione;
	\item \texttt{nrobots} il numero di robot che devono esplorare l'area di interesse;
	\item \texttt{ncells} la lunghezza, in termini di celle, del lato del quadrato che rappresenta il territorio, si è adottata una rappresentazione di un ambiente quadrato per comodità ma tutti i risultati sono estendibili a mappe rettangolari; si sottolinea, inoltre, che tale parametro non tiene contro della “cornice”, questa viene aggiunta in un secondo momento in maniera trasparente all'utente;
	\item \texttt{obstacles\_dist} indica la probabilità con cui ogni singola cella possa essere un ostacolo e quindi inesplorabile, questo valore risulta determinante nel momento in cui non si utilizzino delle mappe pregenerate;
	\item \texttt{wifi\_range} indica la lunghezza del raggio del singolo ripetitore \textit{wi-fi} in termini di celle, si considera coperta l'area stabilita dal vicinato di Moore di distanza pari al parametro sopracitato;
	\item \texttt{ninjured}, ovvero il numero di feriti all'interno della mappa.
\end{itemize}
\todo[inline]{da rivedere sicuramente quanto segue perché molto ci giochiamo qui DP}
Al contempo, come già detto, all'interno della classe che rappresenta l'ambiente sono stati inseriti un insieme di parametri che vengono sfruttati poi dai singoli agenti, a livello programmativo, ma che sono condivisi, o perché sono valori costanti e immutabili o perché li condividono mediante dei sistemi di comunicazione.
A livello teorico, tali parametri dovrebbero essere personali e rappresentati in ogni singolo agente ma per motivi di memoria e comodità per la parte simulativa sono stati inseriti come parametri della classe che rappresenta l'ambiente poiché tale oggetto è condiviso, per imposizione della libreria, ad essere presente,quindi condiviso, tra i vari agenti.
Di seguito sono riportati:
\begin{itemize}
	\item \texttt{radar\_radius} ovvero la capacità di percezione che ha il robot in termini di quante celle vede di fronte a lui;
	\item \texttt{alpha} e \texttt{gamma} sono due pesi che vengono sfruttati dai robot rispettivamente per la scelta della cella da esplorare e per la diminuzione dell'utilità delle celle, tali parametri verranno poi ampiamente discussi;
	\item \texttt{frontier} rappresenta la frontiera delle celle da esplorare, per frontiera si intende l'insieme delle celle non esplorate che sono adiacenti ad una cella che sta venendo esplorata o che è stata esplorata; tale rappresentazione è costantemente aggiornata tra tutti i robot e ogni robot conosce la frontiera completamente, anche quelle celle che non ha mai individuato in personalmente;
	\item \texttt{seen\_graph} è un grafo diretto in cui ogni nodo rappresenta una cella vista da almeno un robot (ovvero: celle esplorate, di frontiera e celle che sono state individuate ma ancora “distanti” per essere considerate appartenenti alla frontiera) e gli archi sono pesati con il costo per moversi tra le due celle; anche in questo caso, come precedentemente, tale rappresentazione teoricamente è interna ad ogni robot ma rimane sempre aggiornata in maniera concorde alle scelte effettuate dagli altri robot;
	\item \texttt{inj\_pri} stabilisce quale di due tecniche viene utilizzata per aumentare la priorità del vicinato di una cella in cui un ferito ha segnalato la sua presenza.
\end{itemize}
\section{Agenti}
\label{sec:agents}
\subsection{Robot}
\label{sub:robots}
Gli agenti che modellano il comportamento dei robot sono, di fatto, la componenete principale di tutto il sistema.
Sono i robot, e loro soltanto, a muoversi all'interno del territorio cercando i feriti e costruendo, nel mentre, la rete \textit{mesh}; nonostante abbiano un ruolo così fondamentale, questi agenti sono a tutti gli effetti dei \textit{reflexive agent with internal state}, il cui comportamento viene descritto di seguito.
È bene esplicitare, prima di proseguire, qual'è lo stato interno dell'agente descritto: in questo caso, lo stato interno è dato da un insieme di variabili; in particolare:
\begin{itemize}
	\item \texttt{target\_cell}, ovvero la cella di frontiera a cui il robot è diretto;
	\item \texttt{target\_path} è il cammino minimo che l'agente deve seguire per raggiungere la cella obiettivo dalla posizione attuale;
	\item \texttt{status} serve per indicare se il robot sta esplorando una cella, si sta muovendo, sta scegliendo la prossima cella obiettivo (o sta aspettando che nuove celle si aggiungano alla frontiera) oppure che si è rotto.
\end{itemize}
Per alleggerire la lettura del diagramma di flusso sottostante e per comodità di descrizione dell'agente la casistica del fallimento di un robot verrà descritta di seguito separatamente.
Inoltre, come già detto nella Sezione \ref{sec:environment}, questi agenti presentano un insieme di altre variabili che ne descrivono delle caratteristiche o che vengono sfruttate dall'agente per effettuare delle scelte a livello “microscopico” (\textit{e.g.}, quale cella scegliere come obiettivo); a queste si aggiungono tra ulteriori variaibli di interesse: la posizione in cui si trova l'agente, quella precedente e l'utilità della cella scelta come obiettivo prima che venisse modificata (variabile utilizzata per la gestione dei fallimenti dei robot).
Infine, a livello programmativo, sono presenti un insieme di variaibli atte a simulare il tempo che il robot trascorre per spostarsi da una posizione ad un'altra o per esplorare una cella. 

Durante la sua vita, l'agente valuta il suo stato interno e se sta mantenendo una connessione con la rete \textit{mesh} oppure no, in base a queste due condizioni prende delle decisioni “macroscopiche” su quali azioni effettuare (\textit{e.g.}, decide di rilasciare un ripetitore \textit{wi-fi} oppure continuare a muoversi verso l'obiettivo) come mostrato in Figura \ref{fig:robotworkflow}.
È importante far notare che il processo decisionale non avviene ad ogni \textit{step} della simulazione (cioè ad ogni secondo), ma solo quando lo stato interno del robot subisce delle modifiche.
\begin{figure}
	\centering
	\includegraphics[width=1.0\linewidth]{images/Robot_workflow}
	\caption{Figura che rappresenta il processo decisionale che l'agente robot quando deve stabilire la sua prossima azione, tale processo si basa sullo stato interno del robot e sulla presenza (o assenza) di connessione con la rete \textit{mesh}; si noti, che quello appena descritto non avviene ad ogni step della simulazione ma solo quando lo stato interno dell'agente subisce dei cambiamenti.}
	\label{fig:robotworkflow}
\end{figure}
Per prima cosa, il robot valuta se è ancora all'interno della copertura \textit{wi-fi} poiché, per come è stato definito il metodo di comunicazione e coordinamento dei robot, risulta fondamentale che siano sempre in grado di comunicare tra loro.
Tale controllo risulta essere necessario farlo ogni volta che il robot, di fatto, si sposta di una cella perché è l'unico caso in cui si suppone che l'agente possa perdere la connessione uscendo dall'area coperta; per comodità e “pulizia” algoritmica viene anche effettuato nel caso in cuo il robot ha finito di esplorare una cella.
Quando il robot, muovendosi, esce dalla zona coperta, se ne accorge e allo \textit{step} successivo rientra immeditamente nella zona coperta iniziando il processo di rilascio del \textit{bean}: aggiorna il suo stato in fase di \textit{deploy}, iniziando poi il processo di rilascio il quale abbiamo supposto impieghi circa un 15 secondi (\textit{i.e.}, 15 \textit{step}) poiché il rilascio del ripetitore deve essere effettuato in un luogo e in modo sicuro.\\
Altrimenti, se il robot è in una zona coperta, la sua decisione viene determinata dal possedimento di una cella \textit{target}.
In particolare, se non possiede una cella obiettivo deve sceglierne una: per prima cosa, si mette in stato di \textit{idling}, ovvero lo stao rappresentante che il robot è fermo per calcolare il suo prossimo obiettivo oppure che non ha celle tra cui scegliere.
In seguito deve stabilire la cella “migliore” da esplorare; per far ciò, i robot sfruttano un concetto di \textit{information gain} (o \textit{info-gain}) \todo[inline]{qua mi sa che c'è da citare l'articolo dicendo cosa noi abbiamo cambiato rispetto a loro, puoi farlo te? Grazie DP} che è uno scalare che indica “quanta informazione può portare l'esplorazione di una cella”, ovvero più tale valore è alto più ad un agente conviene andare ad esplorare tale cella.
La Formula \ref{math:info-gain} è quella utilizzata dagli agenti per il calcolo dell'\textit{info-gain}, tale valore viene calcolato per ogni cella della frontiera.
\begin{equation}
	\label{math:info-gain}
	\textit{Information gain} = \rho+\mu-\alpha\omega
\end{equation}
In tale formula:
\begin{itemize}
	\item $\rho$ è la priorità della cella che risulta essere pari a zero tranne nei casi in cui la cella sia adiacente ad una cella in cui una vittima sia riuscita a segnalare la sua presenza;
	\item $\mu$ è l'utilità della cella;
	\item $\alpha$ è il parametro, già nominato in precedenza, che indica quanto il costo del percorso per raggiungere la cella d'interesse pesi nella scelta;
	\item $\omega$ è il costo, in termini di \textit{step} necessari per raggiungere la cella d'interesse.
\end{itemize}
Per il calcolo di $\omega$ per ogni cella della frontiera, viene computato il costo del cammino minimo sfruttando la rappresentazione interna del territorio che possegono (e condividono) i robot sotto forma di grafo; \textit{i.e.}, di fatto si calcolano insieme tutti i cammini minimi (e i loro costi) dalla cella in cui è presente il robot verso tutte le altre celle sfruttando l'algoritmo definito da Dijkstra.
Infine, verrà scelta la cella con \textit{info-gain} maggiore tra tutte le celle della frontiera, diventando così la cella \textit{target} dell'agente.
Se il robot è riuscito ad individuare la sua prossima cella obiettivo, imposta l'utilità di tale cella pari a $-\infty$ in modo che nessun altro agente scelga tale cella e poi \todo[inline]{perché la rimuoviamo in find best cell la cella dalal frotniera? non dovremmo rimuoverla quando il robot la esplora? per come abbiamo definito la frontiera quella cella è ancora in frotniera, come lo sistemiamo nella relazione? DP}.
Vi sono dei casi in cui è possibile che l'agente non riesca a stabilire il suo prossimo obiettivo perché non vi sono celle nella frontiera oppure perché tutte le celle hanno utilità pari a meno infinito e quindi stanno già venendo esplorate da un altro robot; in questi casi, i robot aspettano un secondo prima di riaggiornare la rappresentazione del territorio condivisa e poi cercano nuovamente una possibile cella.
% riduzione dell'utilità attorno alla cella scelta -> gamma
\todo[inline]{Anche questa riduzione non sarebbe da fare una volta che il robot arriva sulla cella e percepisce attorno? DP}
\todo[inline]]{Perché proprio diviso il radar radius? Puoi aggiungerlo te? Grazie DP}
% parametri
% metodo di comunicazione tra i robot
% metodologie di prioritizzazione
\todo[inline]{fallimenti parlarne altrove? DP}
\subsection{Ferito}
Questo agente rappresenta le persone ferite che sono disperse nell'area di interesse e necessitano di essere individuate e salvate, si noti che il loro salvataggio non fa parte di questo simulatore poiché le metodologie di salvataggio dipendono da troppi fattori che non possono venir considerati complessivamente in un simulatore (\textit{e.g.}, la loro possibilità di muoversi oppure se richiedono un intervento medico sul campo).
Il loro comportamento è stocastico ma basato su uno stato interno, di conseguenza tali agenti sono stati classificati come \textit{reflexive agent with internal state}, facendo riferimento alla classificazione proposta da Russell e Norvig \cite{russell2016}.
L'agente è molto semplice, possiede due attributi che lo descrivono: la posizione all'interno della griglia e lo stato interno; lo stato assume valore 0 nel momento in cui non è ancora stato trovato e 1 quando è stato trovato e quindi salvato successivamente.
Il suo comportamento, come detto, dipende dal suo stato interno, in particolare: 
\begin{itemize}
	\item finché tale agente non è stato individuato e la cella in cui si trova non è coperta dal \textit{wi-fi} non fa nulla;
	\item se si trova in una cella coperta e non è ancora stato individuato, ha una probabilità pari a $10^{-3}$ di segnalare la sua presenza collegandosi alla rete \textit{mesh}, tale probabilità risulta essere bassa perché bisogna considerare che ad ogni \textit{step} (un secondo di tempo d'orologio) della simulazione un ferito può segnalare la sua presenza ed inoltre non tutte le persone potrebbero aver accesso ad un telefono in una situazione critica;
	\item se è stato individuato da un robot o ha già segnalato la sua presenza, non fa altro che aspettare.
\end{itemize}
% Le due metodologie di prioritizzazione dei feriti le nasconderei sotto il cappuccio e ne parlerei nei robot DP