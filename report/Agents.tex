\subsection{Robot}
\label{sub:robots}
Gli agenti che modellano il comportamento dei robot sono, di fatto, la componenete principale di tutto il sistema.
Infatti, sono loro, e loro soltanto, a muoversi all'interno del territorio cercando i feriti e costruendo, nel mentre, la rete \textit{mesh}; nonostante abbiano un ruolo così fondamentale, questi agenti sono a tutti gli effetti dei \textit{reflexive agent with internal state}, il cui comportamento verrà descritto di seguito.

% metodo di comunicazione tra i robot
% metodologie di prioritizzazione
\todo[inline]{fallimenti parlarne altrove? DP}
\subsection{Ferito}
Questo agente rappresenta le persone ferite che sono disperse nell'area di interesse e necessitano di essere individuate e salvate, si noti che il loro salvataggio non fa parte di questo simulatore poiché le metodologie di salvataggio dipendono da troppi fattori che non possono venir considerati complessivamente in un simulatore (\textit{e.g.}, la loro possibilità di muoversi oppure se richiedono un intervento medico sul campo).
Il loro comportamento è stocastico ma basato su uno stato interno, di conseguenza tali agenti sono stati classificati come \textit{reflexive agent with internal state}, facendo riferimento alla classificazione proposta da Russell e Norvig \cite{russell2016}.
L'agente è molto semplice, possiede due attributi che lo descrivono: la posizione all'interno della griglia e lo stato interno; lo stato assume valore 0 nel momento in cui non è ancora stato trovato e 1 quando è stato trovato e quindi salvato successivamente.
Il suo comportamento, come detto, dipende dal suo stato interno, in particolare: 
\begin{itemize}
	\item finché tale agente non è stato individuato e la cella in cui si trova non è coperta dal \textit{wi-fi} non fa nulla;
	\item se si trova in una cella coperta e non è ancora stato individuato, ha una probabilità pari a $10^{-3}$ di segnalare la sua presenza collegandosi alla rete \textit{mesh}, tale probabilità risulta essere bassa perché bisogna considerare che ad ogni \textit{step} (un secondo di tempo d'orologio) della simulazione un ferito può segnalare la sua presenza ed inoltre non tutte le persone potrebbero aver accesso ad un telefono in una situazione critica;
	\item se è stato individuato da un robot o ha già segnalato la sua presenza, non fa altro che aspettare.
\end{itemize}
% Le due metodologie di prioritizzazione dei feriti le nasconderei sotto il cappuccio e ne parlerei nei robot DP