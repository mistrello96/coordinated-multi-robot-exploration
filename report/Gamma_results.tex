Le analisi relative al parametro $\gamma$ sono state effettuate in maniera analoga a quelle precedenti: per ogni valore d'interesse di $\gamma$ sono state effettuate 10 simulazioni ognuna delle quali mantenendo invariati tutti gli altri parametri.
In particolare, le dimensioni della mappa sono state mantenute pari a 30$\times$30 celle, sono stati impiegati 6 robot con un raggio di visione pari a 6 celle, come suggerito dal processo di ottimizzazione descritto nel Capitolo \ref{chap:pso}, un raggio del ripetitore pari a 3 celle e abilitando i feriti a poter segnalare le loro posizioni.
Per quanto riguarda le mappe, sono state sfruttate le 5 mappe generate casualmente e utilizzate per le analisi condotte sul parametro $\alpha$ (Sezione \ref{sec:alpha}).
Infine, poiché ci si è accorti che il valore di $\alpha$ sia in grado di influenzare la distanza che mantengono tra loro i robot, si è deciso di effettuare due batterie di test: una prima in cui si è mantenuto un valore di $\alpha$ basso pari a $10^{-4}$ e poi per uno elevato pari a 8.175 (valore consigliato dal processo di ottimizzazione).
Le due batterie di test sono state eseguite in modo separato e autonomamente l'una dall'altra, non modificando altri parametri (escluso $\gamma$).\\
La metrica utilizzata per valutare come il parametro influisce nel comportamento del modello è stata la distanza media tra i robot. Come già detto, ci si aspetta che per valori alti di $\gamma$ gli agenti tendano a tenersi più distanti tra loro e viceversa.
Formalmente, la metrica viene computata nel seguente modo: per ogni robot si è calcolata la media delle distanze euclidee tra l'agente di interesse e tutti gli altri robot; in seguito, si è computata la media (e relativa deviazione standard) delle distanze medie.
Tale metrica è stata misurata ad ogni \textit{step} della simulazione.
Di seguito, mostriamo i principali risultati ottenuti in entrambe le batterie di test, ulteriori dati prodotti, meno significativi, sono riportati in Appendice \ref{apx:gamma}.

\subsection{Analisi con valore di $\alpha$ basso}
\label{subsec:gammaalow}
Una prima analisi effettuata è stata considerare per ogni valore del parametro la distribuzione della distanza media, ovvero in quanti \textit{step} si è presentato un determinato valore di distanza media; tali distribuzioni sono state calcolando aggregando i dati di tutte le simulazioni per ogni valore del parametro.
In Figura \ref{fig:gammaDistr} sono mostrate le distribuzioni per due valori di $\gamma$ (le altre sono riportate in Appendice \ref{apx:gamma}); si noti che la scala sull'asse delle \textit{y} è logaritmica e la distribuzione è rappresentata solo da un \textit{marker} che indica il numero di \textit{step} in cui si è verificate quel valore di distanza. Si tenga inoltre presente che i valori di distanza sono stati raggruppati per numeri interi.
Nelle distribuzioni mostrate, si nota come che per un valore di $\gamma$ pari a 0.32 (Figura \ref{sfig:gammaDistr0.32}), gli agenti tendano a mantenere una distanza nell'“intorno” di 30, con pochi casi di distanze pari o superiori a 50; al contempo, con un valore pari a 0.65 (Figura \ref{sfig:gammaDistr0.65}), si nota come non vi sia un valore di distanza che viene assunto significativamente più frequentemente di altri, ma il grafico presenta una distribuzione più “liscia”.\\
\begin{figure}
	\subfloat[Distribuzione della distanza media mantenuta dai robot con un valore di $\gamma$ pari a 0.32.\label{sfig:gammaDistr0.32}]{
		\includegraphics[width=.47\linewidth]{images/gamma_results/low_alpha/distribution_distance_gamma_0_32}
	}
	\hfill
	\subfloat[Distribuzione della distanza media mantenuta dai robot con un valore di $\gamma$ pari a 0.65.\label{sfig:gammaDistr0.65}]{
		\includegraphics[width=.47\linewidth]{images/gamma_results/low_alpha/distribution_distance_gamma_0_65}
	}
	\caption{Distribuzione della distanza media mantenuta dai robot al variare del valore di $\gamma$, per valutare la distribuzione si è conteggiato in quanti \textit{step} i robot hanno presentato tale distanza, il numero di \textit{bin} per produrre la distribuzione è pari alla differenza tra la distanza minima e massima presente durante la simulazione, tale valore è stato poi arrotondato. Infine si noti che l'asse delle \textit{y} è a scala logaritmica.}
	\label{fig:gammaDistr}
\end{figure}
Poiché effettuare dei confronti solo mediante grafici di distribuzioni realizzati ognuno per ogni valore analizzato di $\gamma$ risulta complicato e poco significativo, si è deciso di confrontarli unendo tali distribuzioni nello stesso grafico; si noti che l'asse delle \textit{y} non è più a scala logaritmica, in modo da favorire un confronto qualitativo delle distribuzioni.
Il grafico complessivo, riportato in Figura \ref{fig:gammaComparison}, mostra come al crescere de valore $\gamma$ la distanza media tra gli agenti tende ad aumentare per un numero maggiore di \textit{step}.
In particolare si può notare come che per valori di $\gamma$ pari a 0.32 e 1 si presentino dei veri e propri “picchi”; è interessante che per il valore pari a 0.65 non vi sia un vero e proprio “picco” ma, come già detto, la distribuzione risulti più “liscia”, presentando al contempo un numero maggiore di \textit{step} con distanze medie tra 40 e 50 (che risultano essere distanze elevate) e che non compaiono in modo così significativo per nessun altro valore di $\gamma$.
Per valori bassi di $\gamma$ non si evidenziano significative differenze in termini di distanza, ma comunque tende ad emergere una concentrazione di valori di distanza relativamente bassi rispetto agli altri valori di $\gamma$.
In particolare, valutando con attenzione la Formula \ref{math:utility-red} grazie a cui l'utilità di una cella viene ridotta, possiamo intuire che un valore basso di $\gamma$ porta tutte le celle percepite dal robot ad avere un'utilità molto simile e prossima allo $0$. Viceversa, un valore alto di $\gamma$ genera una netta differenza tra celle immediatamente circostanti alla posizione e celle più lontane, garantendo un'efficace meccanismo di dissuasione per gli altri robot.\\
Riassumendo quanto valutato fin'ora, si nota come con un valore di $\alpha$ basso, il parametro $\gamma$ influenza la distanza tra gli agenti, presentando un comportamento particolare per il valore di $\gamma$ pari a 0.65 che è stato quello suggerito dal processo di ottimizzazione (si faccia riferimento al Capitolo \ref{chap:pso}).
\begin{figure}
	\centering
	\includegraphics[width=0.9\linewidth]{images/gamma_results/low_alpha/comparison}
	\caption{Grafico che riassume tutte le distribuzioni di distanze medie durante le simulazioni per ogni valore di $\gamma$, si noti che l'asse delle \textit{y} non è più logaritmico.}
	\label{fig:gammaComparison}
\end{figure}

Infine, \margin{Come $\alpha$ influenza la distanza tra gli agenti}si è andati a studiare l'evolversi all'interno della singola simulazione (ne è stata scelta una casualmente tra le dieci eseguite per ogni valore di $\gamma$ studiato) della distanza media tra i robot.
Tali evoluzioni sono mostrate in Figura \ref{fig:gammaSim} per quanto riguarda i valori del parametro pari a 0.32 (Figura \ref{sfig:gammaSim0.32}) e 0.65 (Figura \ref{sfig:gammaSim0.65}), per gli altri si faccia riferimento all'Appendice \ref{apx:gamma}.
In entrambe le simulazioni emerge come più di una volta la distanza tra gli agenti sia diminuita drasticamente e repentinamente per poi, dopo pochi \textit{step}, ricominciare ad aumentare.
Si è associato tale fenomeno alla possibilità da parte dei feriti di segnalarsi, in particolare, ogni volta che qualcuno si segnala, la priorità delle celle del vicinato viene incrementata, e quindi incrementa l'\textit{info-gain} di tali celle.
Notiamo però che questo fenomeno sembra essere meno presente mano a mano che gamma cresce: questo è probabilmente dovuto al fatto che, al crescere di gamma, l'utilità delle celle nell'intorno di una cella che viene esplorata è fortemente diminuito, mentre le celle più distanti subiscono meno questo malus. Nella scelta della cella bersaglio, essendo $\alpha$ molto basso, la differenza tra utilità e priorità di una cella potrebbe propendere verso la prima componente, rendendo quindi una cella proritizzata vicina a una cella già esplorata meno interessante di una priva di priorità ma più distante.
Poiché il valore di $\alpha$ è basso, i robot non faranno pesare in maniera significativa il costo del cammino e quindi, anche se distanti, tenderanno a scegliere tali celle, portando quindi ad un avvicinamento dei robot con lo scopo di esplorare completamente (e in poco tempo) tutta l'area attorno a dove si è segnalato un ferito.
\begin{figure}
	\subfloat[Evoluzione della distanza media tra gli agenti con un valore di $\gamma$ pari a 0.32.\label{sfig:gammaSim0.32}]{
		\includegraphics[width=.47\linewidth]{images/gamma_results/low_alpha/dinstance_simulation_gamma_0_32_simid_0}
	}
	\hfill
	\subfloat[Evoluzione della distanza media tra gli agenti con un valore di $\gamma$ pari a 0.65.\label{sfig:gammaSim0.65}]{
		\includegraphics[width=.47\linewidth]{images/gamma_results/low_alpha/dinstance_simulation_gamma_0_65_simid_0}
	}
	\caption{Sull'asse delle \textit{x} si trova il tempo, in termini di \textit{step}, impiegato per l'esplorazione della mappa, mentre sull'asse delle \textit{y} la distanza media tra gli agenti.}
	\label{fig:gammaSim}
\end{figure}

\subsection{Analisi con valore di $\alpha$ alto}
\label{subsec:gammaahigh}
Come in precedenza, le prime analisi effettuate si sono concentrate sulla distribuzione delle distanze medie.
In Figura \ref{fig:gammaHDistr} sono rappresentate le due distribuzioni per i valori di $\gamma$ considerati in precedenza; si può notare che con valori alti di $\alpha$, l'andamento della distribuzione si sia invertita rispetto al caso precedente: per un valore pari a 0.32 (Figura \ref{sfig:gammaHDistr0.32}) la distribuzione sembra essere più “liscia”, invece per un valore pari a 0.65 (Figura \ref{sfig:gammaHDistr0.65}) o superiore sembra infittirsi per alcuni valori di distanza.
\begin{figure}
	\subfloat[Distribuzione della distanza media mantenuta dai robot con un valore di $\gamma$ pari a 0.32.\label{sfig:gammaHDistr0.32}]{
		\includegraphics[width=.47\linewidth]{images/gamma_results/high_alpha/distribution_distance_gamma_0_32}
	}
	\hfill
	\subfloat[Distribuzione della distanza media mantenuta dai robot con un valore di $\gamma$ pari a 0.65.\label{sfig:gammaHDistr0.65}]{
		\includegraphics[width=.47\linewidth]{images/gamma_results/high_alpha/distribution_distance_gamma_0_65}
	}
	\caption{Distribuzione della distanza media mantenuta dai robot al variare del valore di $\gamma$, per valutare la distribuzione si è conteggiato in quanti \textit{step} i robot hanno presentato tale distanza, il numero di \textit{bin} per produrre la distribuzione è pari alla differenza tra la distanza minima e massima presente durante la simulazione, tale valore è stato poi arrotondato. Infine si noti che l'asse delle \textit{y} è a scala logaritmica.}
	\label{fig:gammaHDistr}
\end{figure}
Ancora una volta si è deciso di andare a confrontare tutte le distribuzioni dei valori del parametro analizzati, i risultati sono mostrati in Figura \ref{fig:gammaHComparison}.
Come si può notare, al contrario del caso precedente, per valori pari a 0.65 o 1 si notano dei “picchi” significativi e anche per valori più bassi (0.1 e 0.32) si nota come i robot mantengano una distanza media maggiore; tale risultato sembra contraddire quello detto in precedenza.\\
Per analizzare meglio questi risultati e introdurli nel quadro complessivo, bisogna considerare che non è solo $\gamma$ (come avveniva di fatto precedentemente) ad influire sulla scelta della cella obiettivo e degli spostamenti del robot, ma risulta essere vincolante anche il costo del cammino. 
Poiché il parametro $\alpha$ influisce significativamente e con una magnitudo molto maggiore rispetto all'utilità e priorità della cella nella scelta del bersaglio, il parametro $\gamma$ passa in secondo piano e influisce solo in maniera marginale sulla scelta. Ciò va ad inficiare la distanza tra i robot, poiché tali scelte non vengono più effettuate dando grande importanza all'utilità della cella quanto al tempo necessario per raggiungerla.
\begin{figure}
	\centering
	\includegraphics[width=0.9\linewidth]{images/gamma_results/high_alpha/comparison}
	\caption{Grafico che riassume tutte le distribuzioni di distanze medie durante le simulazioni per ogni valore di $\gamma$, si noti che l'asse delle \textit{y} non è più logaritmico.}
	\label{fig:gammaHComparison}
\end{figure}
Come ulteriore conferma di tale affermazione, si può notare come durante la singola simulazione non si presentino più quelle situazioni in cui la distanza tra i robot decresce significativamente a causa della segnalazione da parte dei feriti. Solo durante le fasi finali dell'esplorazione la distanza tra essi alle volte diminuisce.
Al fine di evitare che i feriti vengano di fatto ignorati, è stata implementata una seconda tecnica di prioritizzazione delle celle che, al posto di utilizzare un valore fisso stabilito a priori, utilizza un valore dipendente da $\alpha$; i risultati di tale metodo, discusso nella Sotto-sezione \ref{Ferito} saranno discussi tra breve.
L'evolversi delle distanze è rappresentato in Figura \ref{fig:gammaHSim}
\begin{figure}
	\subfloat[Evoluzione della distanza media tra gli agenti con un valore di $\gamma$ pari a 0.32.]{
		\includegraphics[width=.47\linewidth]{images/gamma_results/high_alpha/dinstance_simulation_gamma_0_32_simid_0}
	}
	\hfill
	\subfloat[Evoluzione della distanza media tra gli agenti con un valore di $\gamma$ pari a 0.65.]{
		\includegraphics[width=.47\linewidth]{images/gamma_results/high_alpha/dinstance_simulation_gamma_0_65_simid_4}
	}
	\caption{Sull'asse delle \textit{x} si trova il tempo, in termini di \textit{step}, impiegato per l'esplorazione della mappa, mentre sull'asse delle \textit{y} la distanza media tra gli agenti.}
	\label{fig:gammaHSim}
\end{figure}
\subsection{Considerazioni conclusive}
In questo breve sunto, si vogliono evidenziare alcune differenze tra come opera il parametro $\gamma$ rispetto al valore di $\alpha$.
In particolare, si è notato come per valori di $\alpha$ bassi è effettivamente il parametro $\gamma$ ad influire sulla distanza tra i robot portando questi a scegliere le celle con utilità maggiore (poiché il peso influisce in maniera poco significativa) e quindi scegliendo le celle viste da meno robot (poiché sono quelle che hanno subito meno riduzioni di utilità).
In aggiunta, si evidenzia come per valori di $\alpha$ bassi i robot tenderanno a muoversi tutti verso la zona segnalata da un ferito nel momento in cui questi segnalano la loro presenza; al contempo, tale effetto sembra più marginale nel momento in cui $\alpha$ aumenta.
Nonostante queste considerazioni, si nota comunque come per valori di $\gamma$ elevati gli agenti, in entrambi i casi, tendono a rimanere più distanti rispetto a valori bassi (si tenga presente che i “picchi” per $\alpha$ elevati sono presenti per valori di distanza minori).
Ancora una volta, tale fenomeno è dovuto al fatto che quando il costo del cammino è maggiormente significativo nel calcolo dell'\textit{info-gain} diventa il principale parametro che delinea la scelta e quindi riducendo l'effetto dell'utilità delle celle e di conseguenza l'influenza del parametro $\gamma$.
\subsection{Metodo di prioritizzazione alternativo}
Al fine di indagare come il metodo di prioritizzazione alternativo, proposto nella Sotto-sezione \ref{subsec:Ferito}, fosse in grado di modificare il comportamento dei robot negli scenari analizzati in precedenza, sono stati ripetuti i medesimi test utilizzando però la forma di priorità che tiene conto del valore di $\alpha$. Questo dovrebbe portare i robot a evitare situazioni in cui l'intera flotta si raduna in un unico punto a seguito della segnalazione della presenza di un ferito (comportamento evidenziato con valori di $\alpha$ bassi).
Come mostrato nella Figura \ref{fig:NgammaLDistr}, i dati sembrano confermare quanto supposto. Non sono più evidenti decrescite repentine che portano i robot a raggrupparsi in una medesima area, come era invece mostrato nell'analisi corrispettiva eseguita con il precedente metodo di prioritizzazione (Figura \ref{fig:gammaSim})
\begin{figure}
	\subfloat[Evoluzione della distanza media tra gli agenti con un valore di $\gamma$ pari a 0.32.]{
		\includegraphics[width=.47\linewidth]{images/new_gamma_results/low_alpha/dinstance_simulation_gamma_0_32}
	}
	\hfill
	\subfloat[Evoluzione della distanza media tra gli agenti con un valore di $\gamma$ pari a 0.65.]{
		\includegraphics[width=.47\linewidth]{images/new_gamma_results/low_alpha/dinstance_simulation_gamma_0_65}
	}
	\caption{Sull'asse delle \textit{x} si trova il tempo, in termini di \textit{step}, impiegato per l'esplorazione della mappa, mentre sull'asse delle \textit{y} la distanza media tra gli agenti.}
	\label{fig:NgammaLDistr}
\end{figure}
Al contempo, analizzando il grafico comparativo delle distribuzioni di distanza tra i robot ad ogni \textit{step}, riportato in Figura \ref{fig:NgammaLComparison}, vediamo come vi sia un'effettiva minore distanza media per valori di $\gamma$ prossimi a zero rispetto agli altri casi. Questo avvalora la tesi già enunciata in precedenza che sostiene che valori di gamma prossimi al valore nullo portano tutte le celle nel vicinato della cella obiettivo, anche quelle molto distanti, ad avere utilità simile e prossima a zero. Ciò porta il costo del cammino minimo verso l'obiettivo a diventare nuovamente protagonista nel calcolo dell'\textit{info-gain}, rendendo vana la tecnica di repulsione tra robot così implementata.
\begin{figure}
	\centering
	\includegraphics[width=0.9\linewidth]{images/new_gamma_results/low_alpha/comparison}
	\caption{Grafico che riassume tutte le distribuzioni di distanze medie durante le simulazioni per ogni valore di $\gamma$, si noti che l'asse delle \textit{y} non è più logaritmico.}
	\label{fig:NgammaLComparison}
\end{figure}
Per quanto concerne i dati ottenuti con valori di $\alpha$ più elevati, i risultati e le conclusioni sono assimilabili a quelle appena tratte per valori del parametro inferiori (tutti i grafici sono disponibili nell'Appendice \ref{apx:gamma}).
