\chapter{Stato dell'arte}
\label{chap:stateofart}
L'esplorazione di una mappa da parte di agenti robotici è materia ampiamente discussa in letteratura scientifica: molti ricercatori hanno avanzato proposte di come coordinare una flotta di piccoli robot con capacità percettive che siano in grado di esplorare una mappa, delineando la presenza di ostacoli o analizzando il contenuto delle aree più rilevanti. Uno dei principali problemi rilevati dalla lettura di questi articoli è l'assenza di un metodo di comunicazione efficace e solido tra tutta la flotta di robot. In \cite{arkin2002line}, per esempio, l'assenza di un metodo di comunicazione globale tra due robot è possibile fintanto che la distanza tra i due sia al di sotto di un certo valore e non vi siano ostacoli di mezzo. Ciò porta ad un comportamento dei robot definito “vagare ancorati”, che si traduce nell'esplorazione di un robot fintanto che esso mantiene la linea visiva e la connessione con il robot precedente; non appena il robot perde il segnale, esso torna sui suoi passi e smette di esplorare, diventando di fatto un nuovo punto con cui è possibile mantenere la linea visiva. Questa tecnica, oltre a permettere l'esplorazione di un unico robot alla volta, rallentando di molto il completamento della ricerca, è molto dispendiosa in termini di numero di robot necessari ad esplorare una mappa. Al fine di ottimizzare questa tecnica, gli autori propongono di aggiungere della conoscenza a priori sulla mappa da esplorare o sulla locazione dell'obiettivo da raggiungere: questo può ottimizzare i tempi di ricerca di un singolo obiettivo, ma i tempi di esplorazione dell'intera mappa rimarranno pressoché invariati. Il vincolo di mantenere tutti i robot all'interno di un range di comunicazione è un vincolo molto forte, che inficia notevolmente le prestazione in caso di esplorazioni di aree estese.\\
Una possibile strategia per superare questo problema è proposta da \cite{de2009role}, dove ai robot, a cui sono associati dei ruoli, è permesso di infrangere il limite del range di comunicazione. In questo studio, gli autori dividono i robot a loro disposizione in due gruppi: esploratori e trasportatori. I primi hanno il compito di esplorare le aree della mappa ancora sconosciute, mentre i secondi hanno il compito di riportare l'informazione al punto di partenza, ove si suppone essere una sorta di comando centrale. L'idea è quella di lasciare che gli esploratori acquisiscano dati su zone inesplorate della mappa, tornino ad un punto di \textit{rendezvous} stabilito precedentemente con un trasportatore e trasferiscano l'informazione ad esso. Una volta fatto, l'esploratore riprende la sua attività di scoperta, mentre il trasportatore torna verso il comando centrale per aggiungere le nuove informazioni. Il principale svantaggio di questa tecnica è la necessità di fissare a priori i ruoli dei robot, limitando di molto il numero di esploratori sul totale dei robot utilizzati. Su mappe di grandi dimensioni, o particolarmente difficili da percorrere, potremmo trovarci nella situazione in cui la maggioranza del tempo è impiegato per attendere che i robot trasportino l'informazione verso il centro di comando piuttosto che esplorando. Sempre nei risultati di \cite{de2009role}, è stato rilevato come un'altra tecnica di esplorazione sembrasse avere ottime performance: quella basata sull'idea di frontiera. Questo concetto, esplicitato in modo esaustivo in \cite{yamauchi1998frontier}, indica l'insieme di celle non ancora esplorate a diretto contatto con celle già scoperte. Mano a mano che i robot selezionano ed esplorano una cella, la frontiera si modifica ed estende i suoi confini. Su quest'idea sono stati ideati diversi approcci esplorativi, che si traducono in differenti strategie di selezione della successiva cella che un robot dovrà esplorare. Un esempio è la strategia utilizzata in \cite{simmons2000coordination}, ove la selezione della successiva cella da esplorare per ogni robot è stabilita da un server centrale: ogni robot costruisce un insieme di “offerte” che rappresentano la valutazione locale del \textit{tradeoff} tra vantaggio nell'esplorare una cella e costo nel raggiungerla. Il server riceve le proposte dei vari robot e, cercando di massimizzare l'utilità complessiva del sistema, assegna ai robot le varie celle. Il grosso dell'elaborazione è eseguita localmente dai robot, valutando le costi e vantaggi, mentre il sistema centrale si occupa di combinare e coordinare l'informazione in maniera efficiente.
La strategia che ci è apparsa più intessente, e che abbiamo preso in esempio per la realizzazione del nostro lavoro, è quella proposta in \cite{burgard2005}, dove il concetto di frontiera è sfruttato per selezionare le prossime celle da esplorare mediante il calcolo di una funzione di \textit{information-gain}.
Ogni robot, dopo aver terminato l'esplorazione di una cella, calcola il guadagno stimato dall'esplorazione di ognuna delle celle della frontiera, scegliendo quella con valore maggiore.
Questa strategia permette di considerare diversi fattori, tra cui la minimizzazione della potenziale sovrapposizione di informazione causata da più robot che esplorano la medesima area. La tecnica proposta disincentiva i robot a convergere nella medesima area in cui è già passato o è presente un altro robot diminuendo l'utilità delle celle nell'intorno del bersaglio degli agenti.
Ai fini di superare i problemi di comunicazione evidenziati dagli articoli sopracitati, abbiamo cercato se fosse possibile creare una rete di comunicazione tra i robot mediante ripetitori che i robot siano in grado di rilasciare sul terreno durante la loro esplorazione: questo permetterebbe di utilizzare la totalità dei robot disponibili per l'effettiva esplorazione, senza doverne dedicare un grande numero alla mera funzione di ripetitori o di trasportatori di informazione. Grazie alla lettura di \cite{yarali2009wireless}, ci siamo resi conto che la soluzione più efficiente ed economica era la creazione di una rete \textit{wifi mesh} che sfruttasse piccoli ripetitori rilasciabili direttamente dai robot. In particolare, il progetto OWL \cite{OWL}, vincitore della competizione \textit{Call for code} di IBM, ha riscosso il nostro interesse. L'idea è quella di sfruttare piccoli ripetitori, della dimensione di un uovo circa, per creare una rete \textit{mesh} che permetta ai robot una comunicazione globale e stabile: qualora un robot dovesse perdere il segnale, gli basterebbe tornare nella posizione precedente e rilasciare un nuovo ripetitore, che estenderebbe la copertura della rete. Ciò permette di ottenere una rete che si estende progressivamente per la mappa, mano a mano che i robot esplorano lo spazio intorno a loro, e garantisce un meccanismo di comunicazione solido e continuo tra i vari robot. Come affermato da \cite{yarali2009wireless}, le reti \textit{mesh} così create si rivelano essere un'ottima soluzione per situazioni successive ad un disastro naturale, dove le convenzionali reti di comunicazione potrebbero essere state danneggiate o avere problemi di sovraccarico 