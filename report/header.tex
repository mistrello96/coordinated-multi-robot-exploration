\documentclass[a4paper,12pt]{report} % modello del documento
% \usepackage[top=1.8cm, bottom=2cm, left=2cm, right=2cm]{geometry} % margini aumentati
\usepackage[toc,page]{appendix}
\usepackage[italian]{babel} % imposta lingua
\usepackage[babel]{csquotes} % imposta lingua
\usepackage[utf8]{inputenc} % imposta lingua
\usepackage[T1]{fontenc} % imposta lingua
\usepackage{amsmath, amssymb, amsfonts} % pachetto per formule
% \usepackage[parfill]{parskip} % non si dovrebbe fare, ma sostituisce le rientranze dei paragrafi con interlinea
\usepackage{listings} % per poter far riconoscere e colorare codice 
\usepackage{subfig}
\usepackage{xcolor} % pacchetto per testo colorato
\usepackage{float} % pachetto per figure, per posizionamento
\usepackage{booktabs} % pacchetto per tabelle
\usepackage{graphicx, wrapfig} % pachetto per tabelle
\usepackage{tcolorbox} % riquadri colorati
\usepackage[Listato]{algorithm} % pseudocodice
\usepackage{algpseudocode} % pseudocodice
\usepackage[hidelinks]{hyperref} % indice e riferimenti cliccabili e senza riquadro rosso
\frenchspacing %spaziatura italiana per accenti
\usepackage[colorinlistoftodos,prependcaption,textsize=tiny]{todonotes} % note TODO
\usepackage{blindtext} %loreipsum
% CONFIGURAZIONE LINK E RIFERIMENTI
\hypersetup{%
	pdfpagemode={UseOutlines},
	bookmarksopen,
	pdfstartview={FitH},
	colorlinks,
	linkcolor={black}, %COLORE DEI RIFERIMENTI AL TESTO
	citecolor={black}, %COLORE DEI RIFERIMENTI ALLE CITAZIONI
	urlcolor={blue} %COLORI DEGLI URL
}

\usepackage{color} %definizione colori
\definecolor{dkgreen}{rgb}{0,0.6,0}
\definecolor{gray}{rgb}{0.5,0.5,0.5}
\definecolor{mauve}{rgb}{0.58,0,0.82}
\lstset{%
	frame=tb,
	language=Python,
	aboveskip=3mm,
	belowskip=3mm,
	showstringspaces=false,
	columns=flexible,
	basicstyle=\ttm,
	numbers=none,
	numberstyle=\tiny\color{gray},
	keywords=[2]{self},
	keywordstyle=\color{deepblue},
	keywordstyle={[2]\color{deepblue}},
	commentstyle=\color{dkgreen},
	stringstyle=\color{mauve},
	breaklines=true,
	breakatwhitespace=true,
	tabsize=3,
	showstringspaces=false
}{}

\newcommand*{\MyMarginNoteFormat}{%
	\scriptsize \bfseries \leavevmode \color{black}%
}
\newcommand{\margin}[1]{%
	\marginpar
	[\raggedleft  \MyMarginNoteFormat #1]%
	{\raggedright \MyMarginNoteFormat #1}%
}


\begin{document} 

\title{Esplorazione di una mappa tramite l'utilizzo di una flotta di robot coordinati in ambito \textit{search and rescue}} 
\author{Matteo Mistri 808097\\ Daniele Maria Papetti 808027}

\maketitle
\tableofcontents

\chapter{Introduzione}
\label{introduction}
Quando una catastrofe naturale colpisce un centro abitato, il primo scopo dei soccorritori è quello di individuare e salvare delle possibili vittime che sono riuscite a sopravvivere al disastro.
Come spesso succede, queste operazioni possono richiedere giorni interi di lavoro, perché molte macerie devono venir rimosse nella speranza di individuare ancora delle persone vive.
Dato che negli ultimi anni si è iniziato ad impiegare i robot per operazioni di esplorazione di un territorio e situazioni di \textit{search and rescue}, in questo lavoro si propone un'estensione di una tecnica proposta e diffusa in letteratura \cite{burgard2005} che si coniuga con una tecnologia sviluppata negli ultimi anni \cite{OWL}.
Lo scopo del lavoro è quello di fornire un sistema altamente autonomo in supporto alle forze di soccorittori, che sia in grado di esplorare l'area colpita mediante una flotta di robot muniti di tecnologie per l'individuazione di feriti (\textit{e.g.}, sensori termici) comunicando tra loro le scelte prese singolarmente.
Brevemente, per far ciò, gli agenti durante l'esplorazione creano una rete \textit{wi-fi mesh} che gli permette di comunicare tra loro e coordinarsi.
Inoltre, tale rete permette a delle possibili vittime munite di \textit{smartphone} di segnalare la loro condizione e posizione mediante il GPS.
Nonostante si parli più volte di coordinazione tra robot, bisogna evidenziare che non vi è un algoritmo centrale che stabilisce il movimento dei robot, ma sono questi a prendere singolarmente delle decisioni in base alla configurazione del sistema propostogli, facendo quindi la scelta ritenuta ottimale.
Il comportamento dei robot è principalmente stabilito dal valore assunto dai parametri interni, che rimangono costanti durante tutta l'esplorazione, e che permettono al robot di effettuare delle scelte su dove muoversi durante l'esplorazione; si capisce che al variare di questi parametri varia il comportamento emergente del sistema.
In quanto ci si trova in una situazione critica, si vorrebbe che il comportamento emergente del sistema sia quello che permetta l'esplorazione dell'area nel minor tempo possibile; per questo motivo, una buona parte del lavoro si è concentrata nell'analisi di questi parametri.\\
In particolare, per questo lavoro ci si è posti i seguenti obiettivi:\begin{itemize}
	\item adattare il modello proposto dall'articolo al caso di studio e realizzare il simulatore;
	\item indurre autonomaticamente i valori ottimi dei parametri per minimizzare i tempi di esplorazione;
	\item valutare l'effetto su aspetti del comportamento emergente al variare del valore dei parametri.
\end{itemize}
Infine, si sottolinea che il simulatore è stato realizzato sfruttando Mesa \cite{Mesa}.
Un \textit{framework} per Python che permette la realizzazione, l'analisi e la visualizzazione di modelli ad agenti; tali agenti possono venir definiti \textit{ad hoc} dall'utente.
\chapter{Descrizione del modello}
\label{chap:modeldesc}
Il sistema che verrà descritto in questo capitolo ha lo scopo di individuare nel minor tempo possibile tutte le persone ferite all'interno di un'area cittadina che si presuppone essere stata colpita da una qualche calamità naturale.
Per far ciò, si desidera che i robot esplorino la mappa nel minor tempo possibile.\\
La soluzione proposta è un adattamento di un noto lavoro in letteratura realizzato da Burgard \textit{et al.} \cite{burgard2005}, a cui si aggiunge un ulteriore capacità e compito dei robot: creare una rete \textit{wi-fi mesh} mediante il posizionamento di ripetitori realizzati \textit{ad hoc} per queste situazioni \textit{citare paper ibm}.
Tale rete possiede principalmente due scopi: \begin{itemize}
	\item permettere ai feriti non ancora individuati di segnalare la loro presenza se sono all'interno della copertura della rete, come proposto nell'articolo;
	\item permettere ai robot di comunicare liberamente tra loro avendo quindi una comunicazione globale e una facile condivisione di informazioni mantenendo una più o meno elevata autonomia decisionale da parte del singolo agente.
\end{itemize} 
In questo capitolo, verrà prima descritto l'ambiente in cui gli agenti operano e cosa contiene da un punto di vista programmativo, verranno poi descritte le due tipologie di agente presenti all'interno del sistema, descrivendone sia il comportamento e, per i robot, anche come ci si aspetta che comunichino tra loro.
\section{Ambiente}
\label{sec:environment}
L'ambiente in cui agiscono gli agenti è una porzione di territorio, tipicamente cittadino e di dimensioni variaibli, dopo che è avvenuto un evento catalogabile come disastro naturale; per questo motivo, porzioni di tale mappa non potranno essere esplorate, poiché inaccessibili e le parti esplorabili non richiederanno tutte lo stesso tempo di esplorazione a causa della complessità di attraversamento del territorio causato dai detriti.
L'ambiente presenta le seguenti caratteristiche:
\begin{itemize}
	\item parzialmente inaccessibile, perché i robot esplorano in modo progressivo il territorio; \textit{i.e.}, all'inizio non conoscono nulla della mappa se non quello che percepiscono con i loro sensori, man mano che l'esplorazione progredisce conoscono sempre una porzione maggiore del territorio fino a conoscere tutto l'ambiente ad esplorazione ultimata;
	\item stocastico, poiché nonostante il comportamento dei robot è ben definito a priori, quello dei feriti (il secondo tipo di agente) e le probabilità di fallimento dei robot o dei ripetirori \textit{wi-fi} sono stabiliti da regole stocastiche;
	\item sequenziale, le azioni dei singoli robot dipendono da quelle che hanno effettuato in precedenza, inoltre i feriti possono segnalare la loro presenza solo se non sono già stati individuati dai robot;
	\item semi-dinamico, i robot e i feriti non agiscono direttamente sull'ambiente modificando il territorio ma possono modificare l'importanza di sue porzioni, che come descritto in seguito, cambieranno il comportamento degli agenti;
	\item discreto, nonostante la combinatoria sia significativa è possibile stabilire a priori tutte le possibili configurazioni che può assumere l'ambiente con gli agenti al suo interno.
\end{itemize}

Come già accennato, l'ambiente è rappresentato come una griglia in cui ogni cella rappresenta un'area di 3$\times$3 metri e ogni \textit{step} della simulazione corrisponde ad un secondo di tempi di orologio.
Inoltre, attorno al territorio da esplorare, vi è un bordo composto da una “cornice di spessore una cella” che rappresenta la porzione di territorio confinante a quello di interesse in cui verranno disposti i robot, e che questi potranno sfruttare per raggiungere più velocemente altre celle all'interno dell'area di interesse.
Ogni cella è descritta da un insieme di attributi:
\begin{itemize}
	\item le sue coordinate all'interno della griglia, per rappresentazione interna del simulatore viene prima esplicitata la colonna e poi la riga;
	\item un intero che varia nell'intervallo $\\left[1, 12\\right]$ che rappresenta una difficoltà simbolica della cella, più tale valore è alto più il robot impiegherà tempo ad esplorarla e ad attraversarla per raggiungere altre celle (un difficoltà elevata può essere data da un numero maggiore di detriti nella zona o a dei muri che costringerebbero il robot ad effettuare a livello microscopico degli aggiramenti);
	\item lo stato della cella, ovvero se non è ancora stata esplorata, se sta venendo esplorata, se non è esplorabile oppure se è una cella della “cornice”;
	\item un valore di priorità, ovvero un parametro che fa aumentare l'importanza della cella favorendola nella scelta della prossima destinazione da parte dei robot, ciò è dovuto al fatto che la cella si trova nel vicinato di una cella in cui il ferito ha segnalato la sua posizione \todo[inline]{in che intervalli di valori varia? DP};
	\item un valore di utilità, inizilizzato ad uno per ogni cella, che viene sfruttato dai robot per scegliere la cella “migliore” (questo concetto viene meglio delineato nella Sotto-sezione \ref{sub:robots});
	\item due booleani che stabiliscono se in tale cella è stato posizionato un ripetitore \textit{wi-fi} oppure se la cella è coperta dal segnale \textit{wi-fi}, si sottolinea che la rappresentazione della copertura della rete è effettuata mediante questa tecnica e che non si sono sfruttati ulteriori agenti rendendo quindi tale rappresentazione a stretto contatto con l'ambiente.
\end{itemize}

Da un punto di vista programmativo, l'ambiente non rappresenta solo l'ambiente in sè ma si preoccupa di far progradire la simulazione, raccogliere i dati di interesse, contenere dei parametri utilizzati dai singoli robot (per comodità di rappresentazione dei dati e di gestione della memoria) e infine di fungere anche come parte della rappresentazione condivisa della mappa da parte dei robot, quest'ultima è discussa in dettaglio nella Sotto-sezione \ref{sub:robots}.
Di seguito, verranno elencati, e brevemente descritti, solo gli attributi che si riferiscono effettivamente all'ambiente:
\begin{itemize}
	\item \texttt{grid} rappresenta la griglia in cui i vari agenti si muoveranno;
	\item \texttt{schedule} rappresenta uno \textit{scheduler} con ordine di attivazione casuale per l'esecuzione dell'azione degli agenti al relativo \textit{step} della simulazione;
	\item \texttt{nrobots} il numero di robot che devono esplorare l'area di interesse;
	\item \texttt{ncells} la lunghezza, in termini di celle, del lato del quadrato che rappresenta il territorio, si è adottata una rappresentazione di un ambiente quadrato per comodità ma tutti i risultati sono estendibili a mappe rettangolari; si sottolinea, inoltre, che tale parametro non tiene contro della “cornice”, questa viene aggiunta in un secondo momento in maniera trasparente all'utente;
	\item \texttt{obstacles\_dist} indica la probabilità con cui ogni singola cella possa essere un ostacolo e quindi inesplorabile, questo valore risulta determinante nel momento in cui non si utilizzino delle mappe pregenerate;
	\item \texttt{wifi\_range} indica la lunghezza del raggio del singolo ripetitore \textit{wi-fi} in termini di celle, si considera coperta l'area stabilita dal vicinato di Moore di distanza pari al parametro sopracitato;
	\item \texttt{ninjured}, ovvero il numero di feriti all'interno della mappa.
\end{itemize}
\todo[inline]{da rivedere sicuramente quanto segue perché molto ci giochiamo qui DP}
Al contempo, come già detto, all'interno della classe che rappresenta l'ambiente sono stati inseriti un insieme di parametri che vengono sfruttati poi dai singoli agenti, a livello programmativo, ma che sono condivisi, o perché sono valori costanti e immutabili o perché li condividono mediante dei sistemi di comunicazione.
A livello teorico, tali parametri dovrebbero essere personali e rappresentati in ogni singolo agente ma per motivi di memoria e comodità per la parte simulativa sono stati inseriti come parametri della classe che rappresenta l'ambiente poiché tale oggetto è condiviso, per imposizione della libreria, ad essere presente,quindi condiviso, tra i vari agenti.
Di seguito sono riportati:
\begin{itemize}
	\item \texttt{radar\_radius} ovvero la capacità di percezione che ha il robot in termini di quante celle vede di fronte a lui;
	\item \texttt{alpha} e \texttt{gamma} sono due pesi che vengono sfruttati dai robot rispettivamente per la scelta della cella da esplorare e per la diminuzione dell'utilità delle celle, tali parametri verranno poi ampiamente discussi;
	\item \texttt{frontier} rappresenta la frontiera delle celle da esplorare, per frontiera si intende l'insieme delle celle non esplorate che sono adiacenti ad una cella che sta venendo esplorata o che è stata esplorata; tale rappresentazione è costantemente aggiornata tra tutti i robot e ogni robot conosce la frontiera completamente, anche quelle celle che non ha mai individuato in personalmente;
	\item \texttt{seen\_graph} è un grafo diretto in cui ogni nodo rappresenta una cella vista da almeno un robot (ovvero: celle esplorate, di frontiera e celle che sono state individuate ma ancora “distanti” per essere considerate appartenenti alla frontiera) e gli archi sono pesati con il costo per moversi tra le due celle; anche in questo caso, come precedentemente, tale rappresentazione teoricamente è interna ad ogni robot ma rimane sempre aggiornata in maniera concorde alle scelte effettuate dagli altri robot;
	\item \texttt{inj\_pri} stabilisce quale di due tecniche viene utilizzata per aumentare la priorità del vicinato di una cella in cui un ferito ha segnalato la sua presenza.
\end{itemize}
\section{Agenti}
\label{sec:agents}
\subsection{Robot}
\label{sub:robots}
Gli agenti che modellano il comportamento dei robot sono, di fatto, la componenete principale di tutto il sistema.
Sono i robot, e loro soltanto, a muoversi all'interno del territorio cercando i feriti e costruendo, nel mentre, la rete \textit{mesh}; nonostante abbiano un ruolo così fondamentale, questi agenti sono a tutti gli effetti dei \textit{reflexive agent with internal state}, il cui comportamento viene descritto di seguito.
È bene esplicitare, prima di proseguire, qual'è lo stato interno dell'agente descritto: in questo caso, lo stato interno è dato da un insieme di variabili; in particolare:
\begin{itemize}
	\item \texttt{target\_cell}, ovvero la cella di frontiera a cui il robot è diretto;
	\item \texttt{target\_path} è il cammino minimo che l'agente deve seguire per raggiungere la cella obiettivo dalla posizione attuale;
	\item \texttt{status} serve per indicare se il robot sta esplorando una cella, si sta muovendo, sta scegliendo la prossima cella obiettivo (o sta aspettando che nuove celle si aggiungano alla frontiera) oppure che si è rotto.
\end{itemize}
Per alleggerire la lettura del diagramma di flusso sottostante e per comodità di descrizione dell'agente la casistica del fallimento di un robot verrà descritta di seguito separatamente.
Inoltre, come già detto nella Sezione \ref{sec:environment}, questi agenti presentano un insieme di altre variabili che ne descrivono delle caratteristiche o che vengono sfruttate dall'agente per effettuare delle scelte a livello “microscopico” (\textit{e.g.}, quale cella scegliere come obiettivo); a queste si aggiungono tra ulteriori variaibli di interesse: la posizione in cui si trova l'agente, quella precedente e l'utilità della cella scelta come obiettivo prima che venisse modificata (variabile utilizzata per la gestione dei fallimenti dei robot).
Infine, a livello programmativo, sono presenti un insieme di variaibli atte a simulare il tempo che il robot trascorre per spostarsi da una posizione ad un'altra o per esplorare una cella. 

Durante la sua vita, l'agente valuta il suo stato interno e se sta mantenendo una connessione con la rete \textit{mesh} oppure no, in base a queste due condizioni prende delle decisioni “macroscopiche” su quali azioni effettuare (\textit{e.g.}, decide di rilasciare un ripetitore \textit{wi-fi} oppure continuare a muoversi verso l'obiettivo) come mostrato in Figura \ref{fig:robotworkflow}.
È importante far notare che il processo decisionale non avviene ad ogni \textit{step} della simulazione (cioè ad ogni secondo), ma solo quando lo stato interno del robot subisce delle modifiche.
\begin{figure}
	\centering
	\includegraphics[width=1.0\linewidth]{images/Robot_workflow}
	\caption{Figura che rappresenta il processo decisionale che l'agente robot quando deve stabilire la sua prossima azione, tale processo si basa sullo stato interno del robot e sulla presenza (o assenza) di connessione con la rete \textit{mesh}; si noti, che quello appena descritto non avviene ad ogni step della simulazione ma solo quando lo stato interno dell'agente subisce dei cambiamenti.}
	\label{fig:robotworkflow}
\end{figure}
Per prima cosa, il robot valuta se è ancora all'interno della copertura \textit{wi-fi} poiché, per come è stato definito il metodo di comunicazione e coordinamento dei robot, risulta fondamentale che siano sempre in grado di comunicare tra loro.
Tale controllo risulta essere necessario farlo ogni volta che il robot, di fatto, si sposta di una cella perché è l'unico caso in cui si suppone che l'agente possa perdere la connessione uscendo dall'area coperta; per comodità e “pulizia” algoritmica viene anche effettuato nel caso in cuo il robot ha finito di esplorare una cella.
Quando il robot, muovendosi, esce dalla zona coperta, se ne accorge e allo \textit{step} successivo rientra immeditamente nella zona coperta iniziando il processo di rilascio del \textit{bean}: aggiorna il suo stato in fase di \textit{deploy}, iniziando poi il processo di rilascio il quale abbiamo supposto impieghi circa un 15 secondi (\textit{i.e.}, 15 \textit{step}) poiché il rilascio del ripetitore deve essere effettuato in un luogo e in modo sicuro.\\
Altrimenti, se il robot è in una zona coperta, la sua decisione viene determinata dal possedimento di una cella \textit{target}.
In particolare, se non possiede una cella obiettivo deve sceglierne una: per prima cosa, si mette in stato di \textit{idling}, ovvero lo stao rappresentante che il robot è fermo per calcolare il suo prossimo obiettivo oppure che non ha celle tra cui scegliere.
In seguito deve stabilire la cella “migliore” da esplorare; per far ciò, i robot sfruttano un concetto di \textit{information gain} (o \textit{info-gain}) \todo[inline]{qua mi sa che c'è da citare l'articolo dicendo cosa noi abbiamo cambiato rispetto a loro, puoi farlo te? Grazie DP} che è uno scalare che indica “quanta informazione può portare l'esplorazione di una cella”, ovvero più tale valore è alto più ad un agente conviene andare ad esplorare tale cella.
La Formula \ref{math:info-gain} è quella utilizzata dagli agenti per il calcolo dell'\textit{info-gain}, tale valore viene calcolato per ogni cella della frontiera.
\begin{equation}
	\label{math:info-gain}
	\textit{Information gain} = \rho+\mu-\alpha\omega
\end{equation}
In tale formula:
\begin{itemize}
	\item $\rho$ è la priorità della cella che risulta essere pari a zero tranne nei casi in cui la cella sia adiacente ad una cella in cui una vittima sia riuscita a segnalare la sua presenza;
	\item $\mu$ è l'utilità della cella;
	\item $\alpha$ è il parametro, già nominato in precedenza, che indica quanto il costo del percorso per raggiungere la cella d'interesse pesi nella scelta;
	\item $\omega$ è il costo, in termini di \textit{step} necessari per raggiungere la cella d'interesse.
\end{itemize}
Per il calcolo di $\omega$ per ogni cella della frontiera, viene computato il costo del cammino minimo sfruttando la rappresentazione interna del territorio che possegono (e condividono) i robot sotto forma di grafo; \textit{i.e.}, di fatto si calcolano insieme tutti i cammini minimi (e i loro costi) dalla cella in cui è presente il robot verso tutte le altre celle sfruttando l'algoritmo definito da Dijkstra.
Infine, verrà scelta la cella con \textit{info-gain} maggiore tra tutte le celle della frontiera, diventando così la cella \textit{target} dell'agente.
Se il robot è riuscito ad individuare la sua prossima cella obiettivo, imposta l'utilità di tale cella pari a $-\infty$ in modo che nessun altro agente scelga tale cella e poi \todo[inline]{perché la rimuoviamo in find best cell la cella dalal frotniera? non dovremmo rimuoverla quando il robot la esplora? per come abbiamo definito la frontiera quella cella è ancora in frotniera, come lo sistemiamo nella relazione? DP}.
Vi sono dei casi in cui è possibile che l'agente non riesca a stabilire il suo prossimo obiettivo perché non vi sono celle nella frontiera oppure perché tutte le celle hanno utilità pari a meno infinito e quindi stanno già venendo esplorate da un altro robot; in questi casi, i robot aspettano un secondo prima di riaggiornare la rappresentazione del territorio condivisa e poi cercano nuovamente una possibile cella.
% riduzione dell'utilità attorno alla cella scelta -> gamma
\todo[inline]{Anche questa riduzione non sarebbe da fare una volta che il robot arriva sulla cella e percepisce attorno? DP}
\todo[inline]]{Perché proprio diviso il radar radius? Puoi aggiungerlo te? Grazie DP}
% parametri
% metodo di comunicazione tra i robot
% metodologie di prioritizzazione
\todo[inline]{fallimenti parlarne altrove? DP}
\subsection{Ferito}
Questo agente rappresenta le persone ferite che sono disperse nell'area di interesse e necessitano di essere individuate e salvate, si noti che il loro salvataggio non fa parte di questo simulatore poiché le metodologie di salvataggio dipendono da troppi fattori che non possono venir considerati complessivamente in un simulatore (\textit{e.g.}, la loro possibilità di muoversi oppure se richiedono un intervento medico sul campo).
Il loro comportamento è stocastico ma basato su uno stato interno, di conseguenza tali agenti sono stati classificati come \textit{reflexive agent with internal state}, facendo riferimento alla classificazione proposta da Russell e Norvig \cite{russell2016}.
L'agente è molto semplice, possiede due attributi che lo descrivono: la posizione all'interno della griglia e lo stato interno; lo stato assume valore 0 nel momento in cui non è ancora stato trovato e 1 quando è stato trovato e quindi salvato successivamente.
Il suo comportamento, come detto, dipende dal suo stato interno, in particolare: 
\begin{itemize}
	\item finché tale agente non è stato individuato e la cella in cui si trova non è coperta dal \textit{wi-fi} non fa nulla;
	\item se si trova in una cella coperta e non è ancora stato individuato, ha una probabilità pari a $10^{-3}$ di segnalare la sua presenza collegandosi alla rete \textit{mesh}, tale probabilità risulta essere bassa perché bisogna considerare che ad ogni \textit{step} (un secondo di tempo d'orologio) della simulazione un ferito può segnalare la sua presenza ed inoltre non tutte le persone potrebbero aver accesso ad un telefono in una situazione critica;
	\item se è stato individuato da un robot o ha già segnalato la sua presenza, non fa altro che aspettare.
\end{itemize}
% Le due metodologie di prioritizzazione dei feriti le nasconderei sotto il cappuccio e ne parlerei nei robot DP
\chapter{Inferenza dei parametri $\alpha$ e $\gamma$ mediante un processo di ottimizzazione}
\label{chap:pso}
\chapter{Risultati}
\label{chap:results}
come $\alpha$ influisce nel costo del cammino scelto durante i vari \textit{step} della simulazione e come varia il costo medio durante tutta la simulazione al variare del parametro\\
come il variare di $\gamma$ influisce la distanza (euclidea) tra i robot, sia con un valore di $\alpha$ elevato sia con uno basso\\
\\
Macroscopiche:\\
mappa fissa, come il numero di robot influisce la quantità di ripetitori rilasciati\\
Come le dimensioni della mappa influiscono il numero di bean rilasciati (raggio pari a 3)\\
Come la difficoltà totale influisce il numero di step richiesto\\
Mappa fissa(?), come numero di robot influisce fitness\\
\\
Il numero dei robot in uno stato al durante l'esplorazione (robot\_status)\\
\section{Analisi macroscopiche}
\label{sec:results-macro}
Le seguenti analisi hanno lo scopo di rilevare delle possibli relazioni di come alcune caratterisitche (o parametri) del sistema ne possano influire degli altri.
Per effettuare tali analisi, poiché si tratta di un sistema complesso, è stato necessario effettuare un insieme di simulazioni in modo tale da raccogliere i dati che derivano dal comportamento emergente.
Poihé in questa Sezione si presentano analisi differenti tra loro, ogni volta verrà descritta la metodologia utilizzata per la raccolta dati.
Si tenga presente, che il metodo di prioritizzazione delle celle dovuto alla segnalazione di feriti segue la strategia che attribuisce una bassa priorità.\\
Gli studi effettuati sono i seguenti:\begin{itemize}
	\item come il numero dei robot influisce il numero di ripetitori rilasciati e come fa variare il valore di fitness;
	\item come la dimensione della mappa influisce il numero di ripetitori che devono venir rilasciati dagli agenti;
	\item come la difficoltà totale della mappa influenza il numero di step richiesto per completare l'esplorazione.
\end{itemize}
\subsection{Studio sul numero di robot}
\label{subsec:nrobots}
Come già detto, sul variare del numero dei robot sono state effettuate due analisi riguardo due aspetti diversi del sistema.
In entrambi i casi, però, la metodologia di raccolta dati è stata analoga: per tutte le simulazioni effettuate si è mantenuta la stessa mappa \todo{siamo sicuri che anche per la fitness la mappa fosse fissa? DP}di dimensioni 30$\times$30 di cui ci si è garantiti che non presentasse al suo interno casi patologici (\textit{e.g.}, zone inesplorabili poiché irraggiungibli).
Inoltre, i valori di $\alpha$, $\gamma$ e per il raggio dei sensori sono stati utilizzati quelli inferiti dal processo di ottimizzazione, descritto nel Capitolo \ref{chap:pso}, il \textit{range} del ripetitore è stato mantenuto pari a 3 celle in modo da avere risultati scalati correttamente rispetto alle dimensioni che erano state considerate originariamente (mappe 333$\times$333); infine, per ogni valore del numero di robot sono state effettuate 10 simulazioni e i feriti potevano segnalare la loro presenza \todo{confermi? DP}.

Per quanto concerne \margin{Numero di robot \textit{vs} ripetitori rilasciati}il numero di ripetitori rilasciati, come mostrato in Figura \ref{fig:robotsbeans} tale numero aumenta all'aumentare dei robot impiegati nonostante le dimensioni della mappa rimangono costanti.
Tale comportamento può essere considerato ragionevole perché all'aumentare del numero dei robot aumentano i casi in cui due agenti escono contemporaneamente dalla zona coperta e quindi rilasciano due ripetitori la cui area coperta di uno dei due sarebbe bastata per coprire anche parzialmente l'area dell'altro.
Di fatto, i robot non hanno alcun meccanismo di coordinamento per la creazione della rete, ognuna pensa a sé e quindi all'aumentare del numero di robot aumentano i casi in cui due robot rilasciano due ripetitori vicini tra loro non ottimizzando l'utilizzo del singolo ripetitore, portando quindi a molte porzioni di territorio coperte da più di un singolo ripetitore.
Si fa presente, che tale numero non è distorto dal fatto che ogni singolo robot rilascia un ripetitore appena viene disposto sul campo, poiché i robot rilasciano il ripetitore solo se in quella cella non vi è già presente un ripetitore.
\begin{figure}
	\centering
	\includegraphics[width=0.9\linewidth]{images/macro_results/robots_beans}
	\caption{In Figura è rappresentato come il numero di robot influisce il numero di ripetitori rilasciati nel territorio. Sull'asse delle \textit{x} è presente il numero di robot e sull'asse delle \textit{y} il numero di ripetitori disposti. Sono state effettuate 10 simulazioni per ogni valore di numero di robot, in nero è presente la media del numero di ripetitori rilasciati e l'area rossa indica la deviazione standard.}
	\label{fig:robotsbeans}
\end{figure}

Invece, rispetto alla seconda analisi effettuata, si evidenzia che il valore di fitness è quello definito nella Formula \ref{math:pso}.
Per quanto riguarda i risultati ottenuti mostrati in Figura \ref{fig:fitness} si nota come con un numero di robot basso la fitness diminuisce velocemente perché il tempo di esplorazione del territorio diminuisce significativamente; al contrario, con un numero elevato di agenti il valore di fitness diminuisce molto più lentamente perché il tempo di esplorazione non subisce diminuzioni significative e nel mentre vi è un numero sempre maggiore di robot in stato di \textit{idling} mentre ci si approccia al termine dell'esplorazione.
Si fa presente, che i valori rappresentati dai segni tondi in rosso sono il valore medio di fitness che si ha ottenuto con tale numero di robot, invece, le barre nere (poco visibili) indicano la variazione data dalla deviazione standard; il fatto che spesso non si vedano è dato da fatto che la deviazione standard è molto bassa.
Considerando il grafico mostrato e concentrandosi sull'intervallo tra i 51 e 76 robot, ci si pone una domanda aperta a cui servirebbero analisi di costi successive e non fattibili nei termini di questo lavoro: “vale la pena utilizzare 10, 15 o addirituttra 25 robot in più se il valore di fitness e il tempo di esplorazione diminuiscono poco significativamente?”.\\
Pensiamo che per rispondere a questa domanda (escludendo i dibattiti etici a riguardo) sia necessario fare delle analisi tenendo conto il costo di realizzazione dei robot e di quanti robot si hanno a disposizione nella pratica e se sia necessario intervenire su un solo quartiere o più quartieri.
\begin{figure}
	\centering
	\includegraphics[width=0.9\linewidth]{images/macro_results/fitness}
	\caption{Sull'asse delle \textit{x} è indicato il numero di robot, su quello delle \textit{y} il valore di fitness. Poiché per ogni valore sono state effettuate 10 simulazioni, i valori rappresentati sono la media invece in nero sono le barre di errore date dalla deviazione standard, risultano poco visibili poiché i valori di deviazione standard sono molto piccoli rispetto alle scale del grafico.}
	\label{fig:fitness}
\end{figure}

\subsection{Dimensioni della mappa \textit{vs} numero di ripetitori rilasciati}
\todo[inline]{Non ho trovato il csv che ha generato tale immaigne, quindi controlla tutto quello che ho detto, soprattutto che abbiamo fatto 10 simulazioni per ogni dimensioni. Devi esserne certo, non andare a memroia DP}
Per la produzione di questa analisi, sono state effettuate 10 simulazioni per ogni dimensione della mappa considerata, le mappe sono state generate casualmente ad ogni simulazione, i parametri sono stati utilizzati quelli inferiti dal processo di ottimizzazione (si veda il Capitolo \ref{chap:pso}), il raggio dei ripetitori pari a 3 celle e i feriti erano in grado di poter segnalare la loro presenza. \todo{il numero di robot quant'era? DP}.
I risultati mostrati in Figura \ref{fig:beans}, dove la linea nera è la media e l'area in rosso rappresenta la deviazione standard, mostrano che all'aumentare della dimensione del territorio il numero di ripetitori aumenta (come ragionevole) in maniera assimilabile qualitativamente ad una crescita quadratica.
Questo risultato è più che ragionevole considerando che l'aumentare della dimensione del lato dalla mappa fa aumentare la dimensione di un quadrato portando quindi ad una crescita quadratica delle dimensioni che si riflette coerentemente nel numero di ripetitori necessari.
È interessante notare, aggregando questo dato a quello presentato nella Sotto-sezione \ref{subsec:nrobots}, che l'aumento del numero di ripetitori disposti viene influenzato maggiormente dal numero di agenti impiegati e dal fatto che non si coordinano tra loro rispetto alle dimensioni effettive della mappa che implica una crescita che è intrinseca con l'aumento delle dimensioni della mappa.
\begin{figure}
	\centering
	\includegraphics[width=0.9\linewidth]{images/macro_results/beans}
	\caption{Figura che rappresenta come all'aumentare delle dimensioni del territorio (sull'asse delle \textit{x} è rappresentata la dimensione del lato della mappa) aumenta il numero di ripetitori (sull'asse delle \textit{y}) necessari per coprire tutta l'area. In nero è rappresentato il valore medio, mentre in rosso la deviazione standard.}
	\label{fig:beans}
\end{figure}

\subsection{Difficoltà della mappa \textit{vs} tempo richiesto}
\todo[inline]{Mi confermi che i dati sono quelli nel csv robot\_step\_difficulty.csv? DP}
Per tale studio, sono state effettuate un totale di 7 simulazioni in cui si è fatto aumentare la dimensione della mappa, e con esso in maniera proporzionale il numero di robot impiegati, in modo tale da far aumentare la difficoltà complessiva; per difficoltà complessiva si intende la somma delle difficoltà di ogni singola cella.
Le mappe sono state generate casualmente e si è deciso di non effettuare più simulazioni per ogni difficoltà per non complicare inutilmente il grafico e il processo di generazione dati in aggiunta a tutto il tempo richiesto da tali simulazioni; inoltre, i risultati ottenuti da queste poche simulazioni sono stati considerati soddisfacenti per delle analisi di massima.\todo{sicuramente da sistemare. DP}
Siamo anche convinti che effettuare analisi di fino su queste quantità richiederebbero di prendere in considerazione un numero evelato di fattori e i risultati prodotti potrebbero non risultare altamente interessanti, soprattutto contando che nella pratica ogni situazione sarà unica e poco prevedibile a priori.\\
Per quanto riguarda i risultati riportati in Figura \ref{fig:difficulty} , si nota che anche in questo caso l'aumento di \textit{step} richiesti per completare l'esplorazione non è lineare rispetto alla difficoltà ma presenta piuttosto un andamento più assimilabile a quello di un logaritmo.
In aggiunta a tale prima analisi non è possibile trarre altre conclusioni significative, servirebbero molte più simulazioni e sarebbe necessario definire una strategia complicata per poter tenere in considerazioni più fattori per proporre un'analisi approfondita di tale relazione.
\begin{figure}
	\centering
	\includegraphics[width=0.9\linewidth]{images/macro_results/difficulty}
	\caption{Sull'asse delle \textit{x} è presente la difficoltà complessiva di esplorazione della mappa e sull'asse delle \textit{y} il numero di \textit{step} effettuati per completare l'esplorazione.}
	\label{fig:difficulty}
\end{figure}

\section{Analisi del parametro $\alpha$}
\section{Analisi del parametro $\gamma$}
\section{Stato dei robot}
\chapter{Conclusioni}
Una possibile ottimizzazione, qualora vi fossero i mezzi disponibili, sarebbe una parziale creazione della rete \textit{wi-fi} mediante il rilascio di ripetitori in zone prestabilite grazie a dei droni aerei.
La creazione di una rete in questo modo permetterebbe la riduzione del numero dei ripetitori ridondanti, ma richiederebbe nuovi mezzi e una pianificazione a priori dello schema di posizionamento dei ripetitori.

\begin{appendices}
	\chapter{Ulteriori risultati su come il parametro $\alpha$ influisce nel costo del cammino}
	\label{apx:alpha}
	In questo appendice, si mostrano brevemente i grafici che non sono stati inseriti direttamente nel Sezione \ref{sec:alpha}, dove sono stati discussi gli effetti che ha il parametro $\alpha$ nella scelta dei costi dei cammini.
In particolare, nelle Figure \ref{figapx:alpha1} e \ref{figapx:alpha2} sono riportati i grafici contenenti i costi dei cammini scelti durante la simulazione (aggregati ogni 100 \textit{step} della simulazione).

\begin{figure}
	\begin{tabular}{cc}
		\subfloat[Effetto di un valore del parametro $\alpha$ pari a $10^{-4}$ nella scelta della cella obiettivo in base al costo del cammino.]{\includegraphics[width = .5\textwidth]{images/alpha_results/cost_alpha_0_0001}} &
		\subfloat[Effetto di un valore del parametro $\alpha$ pari a 0.0005 nella scelta della cella obiettivo in base al costo del cammino.]{\includegraphics[width = .5\textwidth]{images/alpha_results/cost_alpha_0_0005}}\\
		\subfloat[Effetto di un valore del parametro $\alpha$ pari a 0.001 nella scelta della cella obiettivo in base al costo del cammino.]{\includegraphics[width = .5\textwidth]{images/alpha_results/cost_alpha_0_001}} &
		\subfloat[Effetto di un valore del parametro $\alpha$ pari a 0.005 nella scelta della cella obiettivo in base al costo del cammino.]{\includegraphics[width = .5\textwidth]{images/alpha_results/cost_alpha_0_005}}\\
	\end{tabular}
	\caption{In colore si denota la media dei costi dei cammini per raggiungere la cella scelta, i valori sono stati calcolati aggregando le scelte effettuate durante intervalli di 100 \textit{step}; in nero gli \textit{errorbar} relativi alla media determinati dalla deviazione standard.}
	\label{figapx:alpha1}
\end{figure}

\begin{figure}
	\begin{tabular}{cc}
		\subfloat[Effetto di un valore del parametro $\alpha$ pari a 0.05 nella scelta della cella obiettivo in base al costo del cammino.\label{sfig:alpha0.05}]{\includegraphics[width = .5\textwidth]{images/alpha_results/cost_alpha_0_05}} &
		\subfloat[Effetto di un valore del parametro $\alpha$ pari a 0.1 nella scelta della cella obiettivo in base al costo del cammino.]{\includegraphics[width = .5\textwidth]{images/alpha_results/cost_alpha_0_1}}\\
	\end{tabular}
		\centering
		\subfloat[Effetto di un valore del parametro $\alpha$ pari a 1 nella scelta della cella obiettivo in base al costo del cammino.]{\includegraphics[width = .5\textwidth]{images/alpha_results/cost_alpha_1}}
	\caption{In colore si denota la media dei costi dei cammini per raggiungere la cella scelta, i valori sono stati calcolati aggregando le scelte effettuate durante intervalli di 100 \textit{step}; in nero gli \textit{errorbar} relativi alla media determinati dalla deviazione standard.}
	\label{figapx:alpha2}
\end{figure}
\end{appendices}

\bibliography{bibliography}
\bibliographystyle{plain}

\end{document}
