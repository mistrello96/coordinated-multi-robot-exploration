\chapter{Conclusioni}
Per lo sviluppo di questo lavoro, dopo aver analizzato lo stato dell'arte, è stato deciso di prendere come riferimento il modello proposto da Burgard \textit{et al.}, estendendolo con ulteriori considerazioni provenienti da altri articoli della letteratura.
Dopo aver definito il modello e i relativi agenti, abbiamo implementato un simulatore che rispecchiasse l'ambiente e la \textit{routine} dei robot, sfruttando il \textit{framework} \texttt{MESA} \cite{Mesa}.
È stato utilizzato un processo di ottimizzazione (FST-PSO \cite{nobile2018}) per inferire una configurazione sub-ottimale dei parametri $\alpha$ e $\gamma$ del simulatore, che influiscono sulle scelte locali prese dai robot e quindi sul comportamento emergente del sistema. 
Sono state effettuate delle analisi macroscopiche sull'andamento delle simulazione che hanno evidenziato come, al crescere del numero di robot utilizzati per esplorare una mappa di configurazione fissa, il valore della funzione di \textit{fitness}, riportata nella  Formula \ref{math:pso}, decresce molto velocemente quando il numero di robot è contenuto, mentre varia in modo marginale per un numero di robot già elevato.
Il tempo necessario per esplorare una mappa non incrementa linearmente rispetto alla difficoltà complessiva della mappa.\\
Uno studio più approfondito eseguito sul parametro $\alpha$ ha rivelato come esso sia in grado di influenzare profondamente il comportamento dei robot nelle scelte locali della successiva cella bersaglio e quindi nel comportamento emergente del sistema.
In particolare, per valori di $\alpha$ prossimi allo zero, i robot sono più proni a scegliere celle con utilità e priorità maggiore, considerando come secondario il costo necessario a raggiungere tale obiettivo. Questo porta ad un drastico deterioramento del tempo totale richiesto per esplorare la mappa (fino a 2 volte superiore).
Al contrario, per valori di $\alpha$ superiori a $10^-2$, il costo per raggiungere la cella obiettivo diventa il metro di scelta prioritario per il robot, portandolo così a scegliere un percorso più economico. Questo porta ad un'esplorazione della terreno più rapida, ma si rischia di perdere l'influenza di utilità e priorità delle varie celle.\\
Per quanto riguarda $\gamma$, abbiamo mostrato come questo parametro influenzi effettivamente il comportamento emergente solo nel caso in cui sia associato a valori di $\alpha$ molto ridotti.
In generale, un valore prossimo allo zero di $\gamma$ genera una situazione di appiattimento dell'utilità delle celle circostanti al robot, con valori di tutte le celle nel raggio di visione prossimi a zero.
Ciò rende di fatto ininfluente l'utilità delle celle nel calcolo dell'\textit{information-gain}, sia con valori di $\alpha$ elevati che ridotti. 
Per valori di $\gamma$ prossimi a uno, i robot tendono a mantenere una distanza reciproca più elevata.\\
L'analisi riguardante lo stato dei robot ha infine evidenziato come la maggior parte del tempo sia impiegato per l'esplorazione delle celle piuttosto che per gli spostamenti o il rilascio dei ripetitori.
Questo indica che l'utilizzo dei parametri inferiti dal processo di ottimizzazione porta i robot a preferire percorsi economici a vantaggio del tempo di esplorazione, facendoci concludere che, al fine di ottenere un'esplorazione il più rapida possibile, il fattore fondamentale da minimizzare è il tempo di spostamento tra le celle.

Concludendo, possiamo considerare raggiunti due dei tre obiettivi che ci siamo posti all'inizio di questo lavoro.
A causa dei tempi computazionali dilatati richiesti dalle singole simulazioni su una mappa di dimensioni 333$\times$333, si è deciso di effettuare il processo di ottimizzazione su una mappa in scala e di ripetere tale processo una sola volta.
Pur consapevoli che in linea teorica un approccio meta-euristico richiede analisi statistiche dei suoi risultati, ci sentiamo di affermare che il risultato ottenuto sia comunque valido, in quanto i valori proposti sono risultati quelli in grado di minimizzare in modo efficace i tempi di esplorazione della mappa da parte dei robot. 
\section{Sviluppi futuri}
Una possibile ottimizzazione, qualora vi fossero i mezzi disponibili, sarebbe una parziale/totale creazione della rete \textit{wi-fi} mediante il rilascio di ripetitori in zone prestabilite grazie a dei droni aerei.
La creazione di una rete in questo modo permetterebbe la riduzione del numero dei ripetitori ridondanti, oltre a rendere possibile una comunicazione precoce della posizione di eventuali feriti, ma richiederebbe nuovi mezzi e una pianificazione a priori dello schema di posizionamento dei ripetitori.\\
Sarebbe inoltre interessante confrontare i tempi di risposta delle forze di soccorso attualmente presenti sul territorio con i risultati ottenuti dai nostri test, in modo da valutare il rapporto costo/vantaggio che un sistema simile al nostro potrebbe apportare in situazioni reali.