L'ambiente in cui agiscono gli agenti è una porzione di territorio, tipicamente cittadino e di dimensioni variaibli, a seguito di un evento catalogabile come disastro naturale; per questo motivo, porzioni di tale mappa non potranno essere esplorate in quanto inaccessibili (\textit{e.g.}, edifici crollati, dislivelli invalicabili). Per la medesima ragione, le aree esplorabili non richiederanno tutte lo stesso tempo di esplorazione, in quanto la complessità di attraversamento del territorio varia a cause dei possibili detriti presenti.
L'ambiente nel suo complesso possiede le seguenti proprietà:
\begin{itemize}
	\item parzialmente inaccessibile, perché i robot esplorano in modo progressivo il territorio; \textit{i.e.}, all'inizio non conoscono nulla della mappa al di fuori di quello che percepiscono con i loro sensori intorno al luogo di dispiegamento. Man mano che l'esplorazione progredisce, i robot conoscono una porzione sempre maggiore del territorio fino a conoscere tutto l'ambiente esplorabile ad esplorazione ultimata;
	\item stocastico, poiché nonostante il comportamento dei robot sia ben definito a priori, quello dei feriti (il secondo tipo di agente) e le probabilità di fallimento dei robot o dei ripetirori \textit{wi-fi} sono stabiliti da regole stocastiche;
	\item sequenziale, le azioni dei singoli robot dipendono dalle azioni e dalle decisioni che hanno effettuato in precedenza. Inoltre i feriti possono segnalare la loro presenza solo se non sono già stati individuati dai robot;
	\item semi-dinamico, i robot e i feriti non agiscono direttamente sull'ambiente modificando il territorio ma possono modificare l'importanza di sue porzioni, che come descritto in seguito, cambieranno il comportamento degli agenti;
	\item discreto, nonostante la combinatoria sia significativa, è possibile stabilire a priori tutte le possibili configurazioni che può assumere l'ambiente con gli agenti al suo interno.
\end{itemize}

Come già accennato, l'ambiente è rappresentato come una griglia in cui ogni cella rappresenta un'area di 3$\times$3 metri e ogni \textit{step} della simulazione corrisponde ad un secondo.
Inoltre, attorno al territorio da esplorare, vi è un bordo composto da una “cornice” di spessore di una cella che rappresenta la porzione di territorio confinante a quello di interesse in cui verranno dispiegati i robot, e che questi potranno sfruttare nella fase di movimento per raggiungere più velocemente altre celle all'interno dell'area di interesse (nel caso in cui il percorso più veloce per arrivare alla cella obiettivo preveda di passare per tali celle).
Ogni cella è descritta da un insieme di attributi:
\begin{itemize}
	\item le sue coordinate all'interno della griglia, per rappresentazione interna del simulatore viene prima esplicitata la colonna e poi la riga;
	\item un intero che varia nell'intervallo $\left[1, 12\right]$ che rappresenta la difficoltà simbolica della cella. Più tale valore è elevato, tanto più i robot impiegheranno ad esplorarla e ad attraversarla per raggiungere altre celle (una difficoltà elevata può essere causata da un numero maggiore di detriti nella zona o dalla presenza di muri che costringerebbero il robot ad effettuare degli aggiramenti a livello “microscopico”);
	\item lo stato della cella, ovvero se è inesplorata, se sta venendo esplorata, se non è esplorabile oppure se è una cella della “cornice”;
	\item un valore di priorità, ovvero un parametro che fa aumentare l'importanza della cella, rendendola più appetibile ai robot nella fase di designazione della prossima destinazione. Questo valore è subordinato al fatto che la cella si trovi nel vicinato di una cella in cui un ferito ha segnalato la propria presenza \todo[inline]{in che intervalli di valori varia? DP};
	\item un valore di utilità, inizializzato ad uno per ogni cella, che viene sfruttato dai robot per scegliere la cella “migliore” nel momento di stabilire il loro prossimo obiettivo durante l'esplorazione (questo concetto viene meglio delineato nella Sotto-sezione \ref{sub:robots}). In particolare, questo valore viene decrementato nel caso in cui una cella sia nelle immediate vicinanze di una cella scelta come target da un robot;
	\item due variabili booleane che stabiliscono se in tale cella è stato posizionato un ripetitore \textit{wi-fi} e se la cella è coperta dal segnale \textit{wi-fi}. Una rappresentazione di questo tipo è stata preferita all'introduzione di un nuovo agente, in modo da rendere la simulazione più efficiente e porre in stretta relazione copertura \textit{wi-fi} e ambiente.
\end{itemize}

Da un punto di vista programmativo, l'ambiente non rappresenta solo la mappa da esplorare, ma si occupa di far progredire la simulazione, raccogliere i dati d'interesse, immagazzinare alcuni parametri globali utilizzati dai singoli robot (per comodità di rappresentazione dei dati e di gestione della memoria). Funge inoltre da rappresentazione condivisa della mappa da parte dei robot (quest'ultima è discussa in dettaglio nella Sotto-sezione \ref{sub:robots}).
Di seguito verranno elencati, e brevemente descritti, solo gli attributi che si riferiscono effettivamente all'ambiente:
\begin{itemize}
	\item \texttt{grid} rappresenta la griglia in cui i vari agenti si muoveranno;
	\item \texttt{schedule} rappresenta uno \textit{scheduler} con ordine di attivazione casuale per l'esecuzione dell'azione degli agenti al relativo \textit{step} della simulazione;
	\item \texttt{nrobots} il numero di robot che devono esplorare l'area di interesse;
	\item \texttt{ncells} la lunghezza, in termini di celle, del lato del quadrato che rappresenta il territorio. È stata adottata una rappresentazione di un ambiente quadrato per comodità, ma tutti i risultati possono essere estesi a mappe rettangolari. Si noti che tale parametro non tiene contro della “cornice”, che viene aggiunta in un secondo momento in maniera trasparente all'utente;
	\item \texttt{obstacles\_dist} indica la probabilità con cui ogni singola cella possa essere un ostacolo e quindi inesplorabile. Questo valore risulta determinante nel momento in cui non vengano utilizzate delle mappe generate in precedenza;
	\item \texttt{wifi\_range} indica la l'ampiezza del raggio di copertura del singolo ripetitore \textit{wi-fi} in termini di celle. Viene considerata coperta l'area inclusa nel vicinato di \textit{Moore} di distanza pari al parametro sopracitato di ogni cella in cui è stato rilasciato un ripetitore;
	\item \texttt{ninjured} rappresenta il numero di feriti all'interno della mappa.
\end{itemize}
\todo[inline]{da rivedere sicuramente quanto segue perché molto ci giochiamo qui DP}
Al contempo, come già accennato in precedenza, all'interno della classe che rappresenta l'ambiente sono stati inseriti un insieme di parametri che vengono sfruttati dai singoli agenti. 
%a livello programmativo, ma che sono condivisi, o perché sono valori costanti e immutabili o perché li condividono mediante dei sistemi di comunicazione.
A livello teorico, tali parametri dovrebbero essere personali e rappresentati in ogni singolo agente, ma per motivi di efficienza programmativa e comodità, nell'implementazione di questo simulatore sono stati inseriti come parametri della classe che rappresenta l'ambiente, poiché tale oggetto è mandatoriamente ed automaticamente condiviso, da parte della libreria, tra i vari agenti.
Di seguito sono riportati tali parametri, con una sintetica spiegazione:
\begin{itemize}
	\item \texttt{radar\_radius} ovvero la capacità di percezione che ha il robot in termini di quante celle vede di fronte a lui;
	\item \texttt{alpha} e \texttt{gamma} sono due pesi che vengono sfruttati dai robot rispettivamente per la scelta della cella da esplorare e per la diminuzione dell'utilità delle celle nell'intorno della cella bersaglio. Tali parametri saranno poi ampiamente discussi;
	\item \texttt{frontier} rappresenta la frontiera delle celle da esplorare. Per frontiera si intende l'insieme delle celle non ancora esplorate che sono adiacenti ad una cella che è stata esplorata completamente o che sta venendo esplorata. Tale rappresentazione è condivisa e costantemente aggiornata tra tutti i robot. Ogni agente conosce completamente la frontiera, anche quelle celle che non ha mai individuato in personalmente;
	\item \texttt{seen\_graph} è un grafo diretto in cui ogni nodo rappresenta una cella vista da almeno un robot (ovvero: celle esplorate, di frontiera e celle che sono state individuate ma ancora “distanti” per essere considerate appartenenti alla frontiera) e gli archi sono pesati con il costo per spostarsi tra le due celle. Anche in questo caso, come per la variabile precedente, tale rappresentazione rimane sempre aggiornata e condivisa tra i vari robot \todo[inline]{in che senso} in maniera concorde alle scelte effettuate dagli altri robot;
	\item \texttt{inj\_pri} stabilisce quale tra due tecniche viene utilizzata per aumentare la priorità del vicinato di una cella in cui un ferito ha segnalato la sua presenza. Tale parametro, che verrà discusso in seguito, permette di scegliere tra una prioritizzazione di valore fisso oppure di valore proporzionale al peso associato al costo del raggiungimento della cellula bersaglio da parte dei robot.
\end{itemize}